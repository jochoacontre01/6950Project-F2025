\documentclass[letterpaper, 10pt]{article}

\usepackage[top=1in, left=1.25in, right=1.25in, bottom=1in]{geometry}
\usepackage[utf8]{inputenc}
\usepackage{graphicx}
\usepackage{fancyhdr}
\usepackage{amsmath}
\usepackage{hyperref}
\usepackage[style=apa, sortcites=true, sorting=nyt, backend=biber]{biblatex}
\usepackage{parskip}
\usepackage{caption}
\usepackage{setspace}
\usepackage{titling}
\usepackage{bookmark}

\captionsetup{font=footnotesize}

\addbibresource{ref.bib}

\author{Jesus Ochoa-Contreras}
\date{October 21, 2025}
\title{IPCC Climate Scenarios: how our decisions will shape our future}

\begin{document}
\spacing{1.0}
\setlength{\parskip}{10pt}
\setlength{\parindent}{30pt}

% \maketitle

\begin{center}
	
	\LARGE{\textbf{IPCC Climate Scenarios: how our decisions are shaping our future}} \\
	\vspace{0.5cm}
	\Large{Course: 6950 - Computer based tools and applications} \\
	\vspace{0.25cm}
	\large{Author: Jesus Ochoa-Contreras} \\
	\vspace{0.25cm}
	\large{\textbf{Fall 2025 - December 4, 2025}} \\

\end{center}


\section{Introduction}

The IPCC (Intergovernmental Panel on Climate Change) was created by the United Nations Environment Programme on 1988, dedicated to provide policymakers with regular scientific assessments on climate change, critical to prepare for potential climate risks and to develop mitigation and adaptation strategies for companies and governments all over the world. The IPCC does not conduct its own research, rather it identifies where there is agreement on the scientific comunity about climate change, reporting it neutrally but staying policy-relevant \parencite{ipcc}.

A climate scenario is an assumption about political changes and their effect on climate change. There are multiple climate scenarios described after the Shared Socio-economic Pathways (SSP), which include plausible socio-political and technological scenarios and its consequences on every variable playing a role on climate change. Under the Coupled Model Intercomparison Project Phase 6 (CMIP6) and through Copernicus Climate Data Store, multiple climate variables are available for free download as gridded data for historic or SSP experiments \parencite{cmip6_ca}.

This project will be focusing on the reduction of snow depth in Newfoundland and Labrador, Canada, since climate change and generalized global warming have imposed increasing stress on the accumulation of snow, crucial for climate patterns. We will explore the relative changes from historical records to assess the climate stress under two main climate scenarios, one with medium carbon emissions and the other with high carbon emissions.

The medium emissions scenario is the Nationally Determined Contributions (NDC, SSP2-4.5), where we assume all countries implement pledged policies in addition to current policies, reaching a peak warming of 2.3°C by 2100. The high emissions scenario is the Current Policies (SSP3-7.0) that assumes only the existing climate policies remain in place but there is not strengthening of ambition level of these policies, reaching a peak warming of 3.0°C \parencite{ngfs_tech}.

\section{Methodology}

The download of the data can be made through two methods: direct download from the webpage, selecting the appropriate parameters, which are the ones in the list below, or through an API call.
\begin{itemize}
	\item Temporal resolution: either daily, monthly, yearly or fixed. Depends on the dataset.
	\item Experiment: one of the multiple available climate scenarios.
	\item Variable: the desired climate variable.
	\item Level: atmospheric height in hPa, may not be available for all variables.
	\item Model: the computational model used to obtain the variable. Not all models contain all variables.
	\item Year: year range for the data, from 1850 to 2300.
	\item Month: month range for the data.
	\item Day: day range for the data.
	\item Geographical area: bounding geographical coordinates of interest.
\end{itemize}

The methodology used in this project is the API call, since it is a programmatic way to easily get data from the database. The webpage offers the exact call we need to make to the API in order to download specific data, and following the pattern we can program a class that allows us to enter the parameters in the form of strings or vectors for ranges and get back the right data. Once downloaded, the data comes in NetCDF format, which is a very efficient format to carry large amounts of data in a relatively small file size. 

The variables obtained through the API call are as follow:
\begin{itemize}
	\item Near-surface air temperature
	\item Snow depth
\end{itemize}

All of the data is downloaded for the climate scenarios called Nationally Determined Contributions (NDC, SSP2-4.5) and Current Policies (SSP3-7.0). The scenario names NDC and Current Policies are based in the nomenclature by the Network for Greening the Financial System \parencite{ngfs_scenarios}, and the codes SSP2-4.5 and SSP3-7.0 are named under the Shared Socioeconomic Pathways as defined in the Intergovernmental Panel on Climate Change sixth assessment in 2023 \parencite{ipcc_report}. These two scenarios are chosen because they are the two more likely scenarios to become reality.

To load such data, we aid ourselves with the Python library \texttt{Xarray}, which is designed to handle large datasets, often with special functions for climate science, while keeping compatibility with numerical and data handling libraries like Numpy and Pandas. Xarray follows more closely the conventions used by Pandas, using as a base object the so-called \texttt{DataArray}, which behaves similar to a Pandas \texttt{DataFrame}. The main difference is that in DataArrays, each point is associated with a set of coordinates, either spatial or time coordinates, so we can retrieve the data for a specific goegraphical point with precision.

It is worth noting that many of the available models are inherently different in how they organize the data inside the files, especially the coordinate naming conventions. For example, the model CESM2 (USA) uses names such as \texttt{lon} and \texttt{lat} for the longitude and latitude coordinates, and vectors ranging from -90 to 90 for latitude and from 0 to 360 for longitude, which are easy to work with. Other models uses an "ij" indexing system, which contains a meshgrid for the coordinates that makes it more difficult to access. That is why we select the model CESM2 (USA) for all variables. 

\begin{figure}
	\centering
	\includegraphics[width=\textwidth]{Fig/CMIP_allModels.png}
	\caption{Results for all 34 available models on the CMIP Yearly Mean Temperature. Each line represents a model}
	\label{fig:all_models}
\end{figure}
Different models give slightly different results for the same variable, and choosing one near the average value is crucial. The model CESM2 (USA) is one of these, visible in Figure \ref{fig:all_models} in a slightly bolder line other than the mean value.

The final data thus is composed of near-surface air temperature and snow depth for the entire world, with a spatial resolution of 1 degree by 1 degree per pixel, monthly for the years 1950-2015 for the historical records and 2015-2100 for the projections. Despite being 2025 as of the writing of this report, the experiments run by the CMIP6 were created during 2016 and has been the last cutoff of modelling. 

There are many ways to visualize this data, but being 2D data, one of the most intuitive is create maps, overlayed on the outline of the countries to provide a frame of reference. The data for the world boundaries can be obtained as an ESRI Shapefile from public datasets \parencite{wbounds} and loaded using the Python library Geopandas, which is especially designed to handle geographical data. The Shapefile format is made of a table of attributes, so it can be thought of a DataFrame with location and geometry attributes, perfectly readable by a library like Geopandas, and loaded to a map to draw the lines of the countries boundaries, which will be the primary basemap.

To create a comparison of the changes across the years on the climate variables we are using, we need to get rid of possible outliers in the data by taking the mean over the winter season. We choose the winter because snow concentration only happens in this season, and including other seasons could bias the results. To select only the winter season, we select the months December, January and February, discard the rest and then take the 5 year average, creating multiple 5-year periods that are easier to compare to one another and eliminating unuseful data in the process. We perform this operation for all the datasets, historical and projected.

To ease the creation of maps for multiple datasets, we create a class that allows us to pass the data and create figures from it afterwards by calling a simple function. The class is called \texttt{PlotObject} and takes an Xarray as input, and the variables and appropriate colormaps or ranges are automatically extracted from the input array.

Apart from creating maps, we can look at the timeseries of the yearly average concentration of snow in the whole area of interest, comparing how it has changed over the years and how it is projected to change under the different climate scenarios. We create line plots that show the trend of these variables, with solid lines and/or dots for the actual data points and dashed lines for polynomial fit on the time series. We chose to perform polynomial fitting to show with more clarity the trend of the future scenarios, where a polynomial of order 5 shows the details of such trend, as a linear or even quadratic fitting would not capture the exact behavior, leaving important trends behind.

An additional insight we can get from the data is that of extreme values, where we try to find what is the first year in the future whose average temperature exceeds the maximum recorded previously. For this, we need to calculate the date corresponding to the maximum recorded temperature, and the average of all the following years. By doing so, we can calculate which year is the first one to break the records consistently.

Finally, to study the relationship between the temperature and the snow depth, we create what is called a regression plot, where we adjust a line using least squares to generate a model that would describe one variable as a function of another, in this case the snow depth as a function of temperature. The expected result is to be an inverse relationship, i.e., the higher the temperature, the lower the snow depth, remembering that we are taking average yearly values in the winter season, where the concentration of snow would be higher.

\section{Results}

\begin{figure}
	\centering
	\includegraphics[width=0.75\textwidth]{Fig/map_snow_hist.png}
	\caption{Historical 5-year average snow depth changes from 1950 to 2015}
	\label{fig:map_snow_hist}
\end{figure}
\begin{figure}
	\centering
	\includegraphics[width=0.75\textwidth]{Fig/map_snow_245.png}
	\caption{Historical 5-year average snow depth changes towards 2100 under the climate scenario SSP2-4.5}
	\label{fig:map_snow_245}
\end{figure}
\begin{figure}
	\centering
	\includegraphics[width=0.75\textwidth]{Fig/map_snow_370.png}
	\caption{5-year average snow depth changes towards 2100 under the climate scenario SSP3-7.0}
	\label{fig:map_snow_370}
\end{figure}

The results of the creation of the maps for the snow depth change over time from a fixed reference point, can be seen in Figures \ref{fig:map_snow_hist}, \ref{fig:map_snow_245} and \ref{fig:map_snow_370}. All maps have been normalized to the maximum and minimum values of the three datasets to ensure comparable visualization between them, however, the lower limit has been capped to increase the visual detail, since in regions like Greenland the snow depth decays so strongly that all other regions would become invisible.

We can see in Figure \ref{fig:map_snow_hist} that the change in snow depth since 1950 has been significant in northern regions like north Quebec and Nunavut, going below 70 cm less than the reference period. In some other regions, like Bonavista, NL, it has seen a slightly increase. 

Comparing the future climate scenarios, in Figure \ref{fig:map_snow_245} we see the changes from the current conditions to 2100 under the medium emissions scenario (SSP2-4.5), where we observe a generalized decrease in all regions except for small areas in Nunavut. Although the difference is not as big as the case in the historical decrease in Figure \ref{fig:map_snow_hist}, we need to remember that this change is on top of the already lost snow depth historically, so we could be looking at reduction of up to 80 cm in snow depth relative to 1950 if we follow the trend of this scenario.

The scenario that the actual trends follow more closely, the high emissions scenario (SSP3-7.0), we see an even more drastic reduction of snow depth all over Canada, visible in Figure \ref{fig:map_snow_370}. Even though the reduction amount is almost the same as that in Figure \ref{fig:map_snow_245}, the reduction is present all over the country, and there are no areas with increases or even the same snow depth we get today. This behavior could have catastrophic consequences for environment and global temperature regulation, but as the scenario name states it, it is where the current policies are taking us.

\begin{figure}
	\centering
	\includegraphics[width=0.75\textwidth]{Fig/snow_change_timeseries.png}
	\caption{Yearly average snow depth in Atlantic Canada, showing the projections of two different climate scenarios.}
	\label{fig:snow_change_timeseries}
\end{figure}

To understand how these scenarios compare to each other more closely, we can look at Figure \ref{fig:snow_change_timeseries}, where the timeseries of the average snow depth during winter on the region in the maps is shown. We see that since around 1980 the trend has been going down drastically, and it is projected to continue that trend for the next 75 years, increasing drastically after 2060 in the high emissions scenario. This could lead to a reduction of 50\% in snow depth during winter time in much of Atlantic Canada.

\begin{figure}
	\centering
	\includegraphics[width=0.75\textwidth]{Fig/map_temp_hist.png}
	\caption{Historical 5-year average temperature changes from 1950 to 2015}
	\label{fig:map_temp_hist}
\end{figure}
\begin{figure}
	\centering
	\includegraphics[width=0.75\textwidth]{Fig/map_temp_245.png}
	\caption{Historical 5-year average temperature changes towards 2100 under the climate scenario SSP2-4.5}
	\label{fig:map_temp_245}
\end{figure}
\begin{figure}
	\centering
	\includegraphics[width=0.75\textwidth]{Fig/map_temp_370.png}
	\caption{5-year average temperature changes towards 2100 under the climate scenario SSP3-7.0}
	\label{fig:map_temp_370}
\end{figure}

Similar to snow depth, we can also look at the changes in temperature in the same time frames and regions, a main driver for the change in snow concentration. We see in Figure \ref{fig:map_temp_hist} that the temperature have already increased in all regions from 1950 to 2015, with more concentration on the regions where we encountered a greater loss of snow in recent years (north Quebec, Nunavut), averaging 1-2 degrees celsius in most regions. When we compare against the medium emissions scenario map in Figure \ref{fig:map_temp_245}, we find that most of the continental area will continue to get warmer by around 4°C from today.

For the high emissions scenario (SSP3-7.0) in Figure \ref{fig:map_temp_370} we can see a very intense warming in all regions, with maximums of up to 7.5°C compared to today's conditions, meaning we could stop seeing below zero temperatures altogether during winter in some regions that have had them historically, plummeting snow depths.

\begin{figure}
	\centering
	\includegraphics[width=0.75\textwidth]{Fig/temp_timeseries.png}
	\caption{Yearly average temperature in Atlantic Canada, showing the projections of two different climate scenarios.}
	\label{fig:temp_change_timeseries}
\end{figure}

The changes in temperature over the years can be seen more clearly in Figure \ref{fig:temp_change_timeseries}, which contains the yearly average temperature for the region in the maps from 1950 to 2015 for observed data, and from 2015 to 2100 for projected data under the two proposed scenarios. We can observe in this figure that the temperature has seen a steady increase since 1950 and follows the same pattern until around 2060 for both scenarios, where we start to see a difference between them. In the case of the high emissions scenario, the temperature starts increasing rapidly until it reaches an average of just above zero degrees; on the other hand, the medium emissions scenario starts to show a slowdown in the rate of rising temperature during the last two decades of the century, attributable to the efforts in reducing carbon emissions to the atmosphere by stricter regulations and actions.

Even though we see in the medium emissions scenario that a reduction of warming rate is possible, the effects are already too strong to ignore, driving an average temperature increase to around 8°C from a century before, and 10°C in the worst-case scenario. Following the trends, we can calculate the year at which the average temperatures are higher than the historical maximum recorded for both scenarios, showing the timframe until or current maximums will be the normal. For the medium emissions scenario, this year is 2048; for the high emissions scenario is 2041, a mere 16 and 23 years. Once we reach this point, each year could become record-breaking.

So far, we have seen a steep decrease in the observed and projected snow depth in much of the area of interest, and a steep increase in the average temperature, from which we can conclude that there is an inverse relationship between those variables, i.e., the higher the temperature, the lower the snow depth. To quantify this relationship, we can resort to a least squares linear fitting to get a mathematical model that allows us to estimate the snow depth as a function of temperature. We create a regression plot using the library Seaborn with all our datapoints on the historical records in the area of interest. 
\begin{figure}
	\centering
	\includegraphics[width=\textwidth]{Fig/regplot2.png}
	\caption{Linear regression plot of snow depth vs temperature. Notice the inverse relationship and snow depth values going to zero when temperatures go above zero}
	\label{fig:regplot2}
\end{figure}

In Figure \ref{fig:regplot2} we can see the results of the linear fitting with all the data points (1 data point per pixel), showing a very clear inverse relationship between the two variables. We can see that as the temperature approaches to zero, so does the snow depth, and on temperatures above 10°C there are no data points for snow depth, meaning snow cannot exist in those temperatures. On the other extreme, we rarely see temperatures lower than -38°C in this specific region.

\section{Conclusion}

Fossil fuels burning over the years has led to a big economic growth and progress in many ways, but it comes with many drawbacks that deserve more attention that we are currently giving them. The concentrations of greenhouse gases in the atmosphere is increasing the global temperature steadily, and what we do to continue or stop doing it can have massive effects on the environment. 

The temperature changes are projected to affect more some areas than others, as seen in Figures \ref{fig:map_temp_hist}, \ref{fig:map_temp_245} and \ref{fig:map_temp_370}, where the continental part of Atlantic Canada is expected to take the biggest hit in temperature from today's conditions, with increases of up to 7.5 degrees celsius during winter. This increase in temperature translates directly into the reduction of ice formation in these regions, and thus a reduction in snow depth. 

Similar to temperature, the snow depth changes are also variable depending on the area, but we see a strong correlation of the greatest reductions in snow depth and the highest temperatures in Figures \ref{fig:map_snow_hist}, \ref{fig:map_snow_245} and \ref{fig:map_snow_370}, where the reduction in the worst-case scenario can be of over 70 cm compared to 70 years prior.

Our findings tell a fact: the current path is taking us to the worst-case scenario. The nonstop increase in temperature (Figure \ref{fig:temp_change_timeseries}) and the alarming decline in snow depth (Figure \ref{fig:snow_change_timeseries}) are warnings that demand immediate response. Even though the goal set by the Paris Agreement to limit global warming by 1.5°C is out of reach, we can still take action to mitigate the worst effects, prompting a system-wide change.

\printbibliography


\end{document}
